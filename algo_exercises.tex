\documentclass[paper=a4, fontsize=11pt]{scrartcl} 
\usepackage[utf8]{inputenc}
\usepackage{amsmath}
\usepackage{amsfonts}
\usepackage{amssymb}
\author{Kim Thuong Ngo}


\usepackage[T1]{fontenc} 
\usepackage{fourier} 

\usepackage{lipsum} 

\usepackage{listings}
\usepackage{graphicx}
\usepackage{tabularx}

\usepackage{sectsty}
\allsectionsfont{\centering \normalfont\scshape} 

\usepackage{fancyhdr} 
\pagestyle{fancyplain} 
\fancyhead{}
\fancyfoot[L]{} 
\fancyfoot[C]{} 
\fancyfoot[R]{\thepage} 
\renewcommand{\headrulewidth}{0pt} 
\renewcommand{\footrulewidth}{0pt}
\setlength{\headheight}{13.6pt}

\numberwithin{equation}{section} 
\numberwithin{figure}{section} 
\numberwithin{table}{section}

\setlength\parindent{0pt} 

\newcommand{\horrule}[1]{\rule{\linewidth}{#1}} 

\title{	
\normalfont \normalsize 
\textsc{Algorithmen} \\ [25pt] 
\horrule{0.5pt} \\[0.4cm] 
\huge Aufgaben \\ 
\horrule{2pt} \\[0.5cm] 
}

\author{Kim Thuong Ngo} 

\date{\normalsize\today} 

%----------------------------------------------------------------------------------------

\begin{document}
\maketitle 
\newpage
\tableofcontents

%----------------------------------------------------------------------------------------
\newpage
\section{Blatt 0}
%----------------------------------------------------------------------------
\subsection{Aufgabe 1: Logarithmus}
\subsubsection*{a)}
Zeigen Sie: $b^{log_{b}(a)} = a$ 

$log_{b}(a) = x \Leftrightarrow b^{x} = a$

Sei $x = log_{b}(a)$ für angemessenes $x \epsilon \mathbb{R}$
$\Rightarrow b^{x} = a$
$\Rightarrow a = b^{x} = b^{log_{b}(a)}$

\subsubsection*{b)}
Zeigen Sie: $log_{b}(x*y) = log_{b}(x) + log_{b}(y)$

Sei $x_{1}=log_{b}(x), x_{2}=log_{b}(y)$
$\Rightarrow b^{x_{1}} = x & b^{x_{2}} = y$
$\Rightarrow x*y = b^{x_{1}}*b^{x_{2}} = b^{x_{1}+x_{2}}$
$\Rightarrow log_{b}(x*y) = x_{1} + x^{2} = log_{b}(x) + log_{b}(y)$

\subsubsection*{c)}
Berechnen Sie $2^{log_{4}(n)}$

$2^{log_{4}(n)}=2^{log_{2}(n^{ \dfrac{1}{2}})} = n^{\dfrac{1}{2}} = \sqrt{n}$

%----------------------------------------------------------------------------
\subsection{Aufgabe 2: Summenformel}
\subsubsection*{a)}
$\sum^{n}_{i=1} i = \dfrac{1}{2} n (n+1)$

\subsubsection*{b)}
$\sum^{n}_{i=1} i^{3} = (\sum^{n}_{i=1} i)^{2}$

\subsubsection*{c)}
$\sum^{n}_{i=1} (i 2^{i}) = 2 + 2^{n+1} (n-1)$

%----------------------------------------------------------------------------
\subsection{Aufgabe 3: Ereignisraum und Ereignisse}
Lösen Sie folgende Aufgaben unter der Annahme, dass Ereignisse gleichverteilt sind.

\subsubsection*{a)}
Bestimmen Sie den Ereignisraum für: "Eine Münze wird drei Mal hintereinander geworfen." Betrachten Sie das Ereignis: "Es wird mindestens zwei Mal Kopf geworfen. " Wie sieht dieses Ereignis als Menge geschrieben aus? Wie ist die Wahrscheinlichkeit für dieses Ereignis?

\subsubsection*{b)}
Zeigen Sie: Aus $A\cap B =\varnothing$ folgt $P[A \cup B] = P[A] + P[B]$. Was gilt, wenn $A \cap B \neq \varnothing$?

\subsubsection*{c)}
Zeigen Sie für das Gegenereignis $A^{C}= \Omega \backslash A$ eines Ergebnisses $A: P [A^{C}] = 1-P[A]$.

%----------------------------------------------------------------------------
\subsection{Aufgabe 4: Zufallsvariable}
Eine Zufallsvariable X ist eine Funktion $X: \Omega \rightarrow M$, wobei $\Omega$ ein Ereignisraum ist und M eine beliebige Menge. \\

Sei $\Omega = {1, ..., 10}^{2}$. Betrachten Sie die Zufallsvariable

$$X: \Omega \rightarrow \mathbb{N}, X(x,y) = x+y.$$

Wir definieren das Ereignis $[X \leq a] = {(x,y) \epsilon \Omega | X(x,y) \leq a}.$

\subsubsection*{a)}
Geben Sie die Menge $[X \leq 5]$ konkret an und beschreiben Sie das Ereignis in Worten.

\subsubsection*{b)}
Berechnen Sie $P[X \leq 5]$ unter Annahme der Gleichverteilung der Ereignisse.

%----------------------------------------------------------------------------
\subsection{Aufgabe 5: Erwartungswert und Varianz}
Sei $\Omega$ ein Ereignisraum. Wir definieren den Erwartungswert einer Zufallsvariable $X: \Omega \rightarrow M$ als

$$E[X] = \sum_{x \epsilon M} x * P[X=x].$$

Intuitiv beschreibt der Erwartungswert einer Zufallsvariable das Ereignis, welches im Mittel am häufigsten auftritt. Der Erwartungswert ist linear, d.h. es gilt

$$E[a+b*X] = a+b * E[X]$$

\subsubsection*{a)}
Berechnen Sie den Erwartungswert einer Zufallsvariable, die nur Werte 0 und 1 haben kann.

\subsubsection*{b)}
Berechnen Sie den Erwartungswert eines fairen Würfels.

\subsubsection*{c)}
Verwenden Sie die Linearität des Erwartungswertes, um den Erwartungswert der Summe von zwei unabhängigen Würfelwürfen zu berechnen.

\subsubsection*{d)}
Die Varianz einer Zufallsvariable X gibt das Mittel der quadratischen Abweichung von X zu ihrem Erwartungswert an. Formal:

$$var(X) = E[(X-E(x))^{2}].$$

Zeigen Sie mit Hilfe der Linearität des Erwartungswerts, dass folgende Gleichung gilt:

$$var(X) = E[X^{2}] - E[X]^{2}.$$

%----------------------------------------------------------------------------------------
\newpage
\section{Blatt 01}
%----------------------------------------------------------------------------
\subsection*{Aufgabe 1: O-Notation}
\subsubsection*{a)}

Aus $f_{1}(n), f_{2}(n) = \mathcal{O}(g(n))$ folgt $f_{1}(n) + f_{2}(n) = \mathcal{O}(g(n))$ und $f_{1}(n) \cdot f_{2}(n) = \mathcal{O}(g(n)^{2})$. \\

\subsubsection*{b)}
Aus $f(n) = \mathcal{O}(g(n))$ und $g(n) = \mathcal{O}(h(n))$ folgt $f(n) = \mathcal{O}(h(n))$. \\

\subsubsection*{c)}
$f(n) = \Theta (g(n))$ genau dann, wenn $g(n) = \Theta (f(n))$. \\

\subsubsection*{d)}
$f(n) = \mathcal{O} (g(n))$ genau dann, wenn $g(n) = \Omega (f(n))$. \\

%----------------------------------------------------------------------------
\subsection*{Aufgabe 2: Mastertheorem}
Bestimmen Sie die Komplexitätsklasse für folgende Rekursionsgleichung mit Hilfe des Mastertheorems:

\subsubsection*{a)}
$T(n) = T( \dfrac{n}{2}) + 1$

\subsubsection*{b)}
$T(n) = 2T( \dfrac{n}{2}) + 1$

\subsubsection*{c)}
$T(n) = 2T( \dfrac{n}{2}) + n$

%----------------------------------------------------------------------------
\subsection*{Aufgabe 3: Rekursionen aus alten Klausuren und geometrische Summenformel}
\subsubsection*{a)}

Zeigen Sie, dass für folgende Rekursion $T(n) = \Theta (n^{2} log n)$ ist. 

$$T(1) = 0$$ 
$$T(n) = T(n-1) + n log n $$

\subsubsection*{b)}
Sei $n = ( \dfrac{8}{7})^{k}$ für ein $k \epsilon \mathbb{N}$. Folgende Rekursion ist für die Funktion T gegeben:

$$T(1)=0$$
$$T(n)= \dfrac{7}{8} T(\dfrac{7}{8} n) + \dfrac{7}{8} n$$

Finden Sie für $T(n)$ eine geschlossene Form ohne das Mastertheorem zu verwenden und beweisen Sie die Korrektheit Ihrer geschlossenen Form mit vollständiger Induktion.

\subsubsection*{c)}
Sei $n = (\dfrac{3}{2})^{k}$ mit $k \epsilon \mathbb{N}$. Folgende Rekursion ist für die Funktion T gegeben:

$$T(1)=0$$
$$T(n)= 2T ( \dfrac{2}{3} n) + 1$$

Finden Sie für $T(n)$ eine geschlossene Form ohne das Mastertheorem zu verwenden und beweisen Sie die Korrektheit Ihrer geschlossenen Form mit vollständiger Induktion.

\subsubsection*{d)}
Sei n eine Zweierpotenz, das heißt $n = 2^{k}$ für ein $k \epsilon \mathbb{N}$. Folgende Rekursion ist für die Funktion T gegeben: Für $n > 1$ gelte

$$T(n) = A(n) + B(n) ,$$ wobei
$$A(n)=A(\dfrac{n}{2}) + B( \dfrac{n}{2})$$ und
$$B(n)=B(n-1)+2n-1.$$

Die Endwerte seien $T(1)=1$, $B(1)=1$ und $A(1)=0$. Finden Sie für $T(n)$ eine geschlossene Form ohne das Mastertheorem zu verwenden und beweisen Sie die Korrektheit Ihrer Lösung.

%----------------------------------------------------------------------------------------


\end{document}