\documentclass[paper=a4, fontsize=11pt]{scrartcl} 
\usepackage[utf8]{inputenc}
\usepackage{amsmath}
\usepackage{amsfonts}
\usepackage{amssymb}
\author{Kim Thuong Ngo}


\usepackage[T1]{fontenc} 
\usepackage{fourier} 

\usepackage{lipsum} 

\usepackage{listings}
\usepackage{graphicx}
\usepackage{tabularx}

\usepackage{sectsty}
\allsectionsfont{\centering \normalfont\scshape} 

\usepackage{fancyhdr} 
\pagestyle{fancyplain} 
\fancyhead{}
\fancyfoot[L]{} 
\fancyfoot[C]{} 
\fancyfoot[R]{\thepage} 
\renewcommand{\headrulewidth}{0pt} 
\renewcommand{\footrulewidth}{0pt}
\setlength{\headheight}{13.6pt}

\numberwithin{equation}{section} 
\numberwithin{figure}{section} 
\numberwithin{table}{section}

\setlength\parindent{0pt} 

\newcommand{\horrule}[1]{\rule{\linewidth}{#1}} 

\title{	
\normalfont \normalsize 
\textsc{Algorithmen} \\ [25pt] 
\horrule{0.5pt} \\[0.4cm] 
\huge Aufgaben \\ 
\horrule{2pt} \\[0.5cm] 
}

\author{Kim Thuong Ngo} 

\date{\normalsize\today} 

%----------------------------------------------------------------------------------------

\begin{document}

\maketitle 

\newpage

\tableofcontents

%----------------------------------------------------------------------------------------

\newpage

\section{Blatt 0}

%----------------------------------------------------------------------------------------

\newpage

\section{Blatt 01}

\subsection*{Aufgabe 1: O-Notation}

\subsubsection*{a)}

Aus $f_{1}(n), f_{2}(n) = \mathcal{O}(g(n))$ folgt $f_{1}(n) + f_{2}(n) = \mathcal{O}(g(n))$ und $f_{1}(n) \cdot f_{2}(n) = \mathcal{O}(g(n)^{2})$. \\

\subsubsection*{b)}

Aus $f(n) = \mathcal{O}(g(n))$ und $g(n) = \mathcal{O}(h(n))$ folgt $f(n) = \mathcal{O}(h(n))$. \\

\subsubsection*{c)}

$f(n) = \Theta (g(n))$ genau dann, wenn $g(n) = \Theta (f(n))$. \\

\subsubsection*{d)}

$f(n) = \mathcal{O} (g(n))$ genau dann, wenn $g(n) = \Omega (f(n))$. \\

%----------------------------------------------------------------------------

\subsection*{Aufgabe 2: Mastertheorem}

Bestimmen Sie die Komplexitätsklasse für folgende Rekursionsgleichung mit Hilfe des Mastertheorems:

\subsubsection*{a)}

$T(n) = T( \dfrac{n}{2}) + 1$

\subsubsection*{b)}

$T(n) = 2T( \dfrac{n}{2}) + 1$

\subsubsection*{c)}

$T(n) = 2T( \dfrac{n}{2}) + n$

%----------------------------------------------------------------------------

\subsection*{Aufgabe 3: Rekursionen aus alten Klausuren und geometrische Summenformel}

\subsubsection*{a)}

Zeigen Sie, dass für folgende Rekursion $T(n) = \Theta (n^{2} log n)$ ist. 

$$T(1) = 0$$ 
$$T(n) = T(n-1) + n log n $$

\subsubsection*{b)}

Sei $n = ( \dfrac{8}{7})^{k}$ für ein $k \epsilon \mathbb{N}$. Folgende Rekursion ist für die Funktion T gegeben:

$$T(1)=0$$
$$T(n)= \dfrac{7}{8} T(\dfrac{7}{8} n) + \dfrac{7}{8} n$$

Finden Sie für $T(n)$ eine geschlossene Form ohne das Mastertheorem zu verwenden und beweisen Sie die Korrektheit Ihrer geschlossenen Form mit vollständiger Induktion.

\subsubsection*{c)}

Sei $n = (\dfrac{3}{2})^{k}$ mit $k \epsilon \mathbb{N}$. Folgende Rekursion ist für die Funktion T gegeben:

$$T(1)=0$$
$$T(n)= 2T ( \dfrac{2}{3} n) + 1$$

Finden Sie für $T(n)$ eine geschlossene Form ohne das Mastertheorem zu verwenden und beweisen Sie die Korrektheit Ihrer geschlossenen Form mit vollständiger Induktion.

\subsubsection*{d)}

Sei n eine Zweierpotenz, das heißt $n = 2^{k}$ für ein $k \epsilon \mathbb{N}$. Folgende Rekursion ist für die Funktion T gegeben: Für $n > 1$ gelte

$$T(n) = A(n) + B(n) ,$$ wobei
$$A(n)=A(\dfrac{n}{2}) + B( \dfrac{n}{2})$$ und
$$B(n)=B(n-1)+2n-1.$$

Die Endwerte seien $T(1)=1$, $B(1)=1$ und $A(1)=0$. Finden Sie für $T(n)$ eine geschlossene Form ohne das Mastertheorem zu verwenden und beweisen Sie die Korrektheit Ihrer Lösung.

%----------------------------------------------------------------------------------------


\end{document}